\documentclass[aspectratio=169]{beamer}

\usetheme{leipzig}

\usepackage{booktabs}
\usepackage{paralist}
\usepackage{blindtext}

\usepackage[ngerman]{babel}

% some macro fun
\setbeamerfont{bibliography item}{size=\tiny}
\setbeamerfont{bibliography entry author}{size=\tiny}
\setbeamerfont{bibliography entry title}{size=\tiny}
\setbeamerfont{bibliography entry location}{size=\tiny}
\setbeamerfont{bibliography entry note}{size=\tiny}

\setbeamerfont{subsection in toc}{size=\footnotesize}

\title[Abschlussarbeiten]{Abschlussarbeiten}
\subtitle[]{Bachelor-/Master-Seminar}
\author{}
\institute{}
\date{
	{\vfill \tiny Letzte Änderung am \today}
}
\titlegraphic{}
\location{}
\thankstitle{}
\address{}
\email{}
\phone{}
\website{https://sws.informatik.uni-leipzig.de}


\pretocmd{\tableofcontents}{\begin{minipage}{\textwidth}}{}{}
	\apptocmd{\tableofcontents}{\end{minipage}}{}{}

\begin{document}
\frame{
	\maketitle
}


\begin{frame}{Übersicht}
  \setbeamertemplate{section in toc}[sections numbered]
  \tableofcontents%[hideallsubsections]
\end{frame}

\section{Ablauf \& Betreuung}
\begin{frame}{Orga}
	Was wie wann?
\end{frame}

\section{Offene Themen}
\subsection{Thesis Topic}
\begin{frame}[allowframebreaks]{Open Thesis Topic}
	\textbf{Ansprechpartner}: \textit{(name of thesis advisor)}
	\begin{compactitem}
		\item \textbf{Kontext}: \textit{(what is the overall context of this thesis topic?)}
		\item \textbf{Motivation} 
		\item \textbf{Problem}: \textit{(somethin)}
		\item \textbf{Scope}: {\textit{scope and expected outcome}}
		\item \textbf{Literatur}: \cite{zeller2011failure,watling2021don} 
	\end{compactitem}
\end{frame}

\section{Laufende Arbeiten}
\input{tex/ongoing_topics.tex}

\section{Literatur}
\begin{frame}[allowframebreaks]{Literatur}
	\setbeamertemplate{bibliography item}[text]
	\bibliographystyle{apalike}
	\bibliography{bibliography.bib}	
\end{frame}

\end{document}
